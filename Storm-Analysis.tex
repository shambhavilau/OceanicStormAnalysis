% Options for packages loaded elsewhere
\PassOptionsToPackage{unicode}{hyperref}
\PassOptionsToPackage{hyphens}{url}
%
\documentclass[
]{article}
\usepackage{amsmath,amssymb}
\usepackage{iftex}
\ifPDFTeX
  \usepackage[T1]{fontenc}
  \usepackage[utf8]{inputenc}
  \usepackage{textcomp} % provide euro and other symbols
\else % if luatex or xetex
  \usepackage{unicode-math} % this also loads fontspec
  \defaultfontfeatures{Scale=MatchLowercase}
  \defaultfontfeatures[\rmfamily]{Ligatures=TeX,Scale=1}
\fi
\usepackage{lmodern}
\ifPDFTeX\else
  % xetex/luatex font selection
\fi
% Use upquote if available, for straight quotes in verbatim environments
\IfFileExists{upquote.sty}{\usepackage{upquote}}{}
\IfFileExists{microtype.sty}{% use microtype if available
  \usepackage[]{microtype}
  \UseMicrotypeSet[protrusion]{basicmath} % disable protrusion for tt fonts
}{}
\makeatletter
\@ifundefined{KOMAClassName}{% if non-KOMA class
  \IfFileExists{parskip.sty}{%
    \usepackage{parskip}
  }{% else
    \setlength{\parindent}{0pt}
    \setlength{\parskip}{6pt plus 2pt minus 1pt}}
}{% if KOMA class
  \KOMAoptions{parskip=half}}
\makeatother
\usepackage{xcolor}
\usepackage[margin=1in]{geometry}
\usepackage{color}
\usepackage{fancyvrb}
\newcommand{\VerbBar}{|}
\newcommand{\VERB}{\Verb[commandchars=\\\{\}]}
\DefineVerbatimEnvironment{Highlighting}{Verbatim}{commandchars=\\\{\}}
% Add ',fontsize=\small' for more characters per line
\usepackage{framed}
\definecolor{shadecolor}{RGB}{248,248,248}
\newenvironment{Shaded}{\begin{snugshade}}{\end{snugshade}}
\newcommand{\AlertTok}[1]{\textcolor[rgb]{0.94,0.16,0.16}{#1}}
\newcommand{\AnnotationTok}[1]{\textcolor[rgb]{0.56,0.35,0.01}{\textbf{\textit{#1}}}}
\newcommand{\AttributeTok}[1]{\textcolor[rgb]{0.13,0.29,0.53}{#1}}
\newcommand{\BaseNTok}[1]{\textcolor[rgb]{0.00,0.00,0.81}{#1}}
\newcommand{\BuiltInTok}[1]{#1}
\newcommand{\CharTok}[1]{\textcolor[rgb]{0.31,0.60,0.02}{#1}}
\newcommand{\CommentTok}[1]{\textcolor[rgb]{0.56,0.35,0.01}{\textit{#1}}}
\newcommand{\CommentVarTok}[1]{\textcolor[rgb]{0.56,0.35,0.01}{\textbf{\textit{#1}}}}
\newcommand{\ConstantTok}[1]{\textcolor[rgb]{0.56,0.35,0.01}{#1}}
\newcommand{\ControlFlowTok}[1]{\textcolor[rgb]{0.13,0.29,0.53}{\textbf{#1}}}
\newcommand{\DataTypeTok}[1]{\textcolor[rgb]{0.13,0.29,0.53}{#1}}
\newcommand{\DecValTok}[1]{\textcolor[rgb]{0.00,0.00,0.81}{#1}}
\newcommand{\DocumentationTok}[1]{\textcolor[rgb]{0.56,0.35,0.01}{\textbf{\textit{#1}}}}
\newcommand{\ErrorTok}[1]{\textcolor[rgb]{0.64,0.00,0.00}{\textbf{#1}}}
\newcommand{\ExtensionTok}[1]{#1}
\newcommand{\FloatTok}[1]{\textcolor[rgb]{0.00,0.00,0.81}{#1}}
\newcommand{\FunctionTok}[1]{\textcolor[rgb]{0.13,0.29,0.53}{\textbf{#1}}}
\newcommand{\ImportTok}[1]{#1}
\newcommand{\InformationTok}[1]{\textcolor[rgb]{0.56,0.35,0.01}{\textbf{\textit{#1}}}}
\newcommand{\KeywordTok}[1]{\textcolor[rgb]{0.13,0.29,0.53}{\textbf{#1}}}
\newcommand{\NormalTok}[1]{#1}
\newcommand{\OperatorTok}[1]{\textcolor[rgb]{0.81,0.36,0.00}{\textbf{#1}}}
\newcommand{\OtherTok}[1]{\textcolor[rgb]{0.56,0.35,0.01}{#1}}
\newcommand{\PreprocessorTok}[1]{\textcolor[rgb]{0.56,0.35,0.01}{\textit{#1}}}
\newcommand{\RegionMarkerTok}[1]{#1}
\newcommand{\SpecialCharTok}[1]{\textcolor[rgb]{0.81,0.36,0.00}{\textbf{#1}}}
\newcommand{\SpecialStringTok}[1]{\textcolor[rgb]{0.31,0.60,0.02}{#1}}
\newcommand{\StringTok}[1]{\textcolor[rgb]{0.31,0.60,0.02}{#1}}
\newcommand{\VariableTok}[1]{\textcolor[rgb]{0.00,0.00,0.00}{#1}}
\newcommand{\VerbatimStringTok}[1]{\textcolor[rgb]{0.31,0.60,0.02}{#1}}
\newcommand{\WarningTok}[1]{\textcolor[rgb]{0.56,0.35,0.01}{\textbf{\textit{#1}}}}
\usepackage{graphicx}
\makeatletter
\def\maxwidth{\ifdim\Gin@nat@width>\linewidth\linewidth\else\Gin@nat@width\fi}
\def\maxheight{\ifdim\Gin@nat@height>\textheight\textheight\else\Gin@nat@height\fi}
\makeatother
% Scale images if necessary, so that they will not overflow the page
% margins by default, and it is still possible to overwrite the defaults
% using explicit options in \includegraphics[width, height, ...]{}
\setkeys{Gin}{width=\maxwidth,height=\maxheight,keepaspectratio}
% Set default figure placement to htbp
\makeatletter
\def\fps@figure{htbp}
\makeatother
\setlength{\emergencystretch}{3em} % prevent overfull lines
\providecommand{\tightlist}{%
  \setlength{\itemsep}{0pt}\setlength{\parskip}{0pt}}
\setcounter{secnumdepth}{-\maxdimen} % remove section numbering
\ifLuaTeX
  \usepackage{selnolig}  % disable illegal ligatures
\fi
\IfFileExists{bookmark.sty}{\usepackage{bookmark}}{\usepackage{hyperref}}
\IfFileExists{xurl.sty}{\usepackage{xurl}}{} % add URL line breaks if available
\urlstyle{same}
\hypersetup{
  hidelinks,
  pdfcreator={LaTeX via pandoc}}

\author{}
\date{\vspace{-2.5em}}

\begin{document}

\hypertarget{u.s.-national-oceanic-and-atmospheric-administrations-storm-data-analysis}{%
\section{U.S. National Oceanic and Atmospheric Administration's Storm
Data
Analysis}\label{u.s.-national-oceanic-and-atmospheric-administrations-storm-data-analysis}}

\hypertarget{overview}{%
\subsection{OVERVIEW}\label{overview}}

Weather events cause public health and economic problems for communities
and municipalities. Severe events result in fatalities, injuries, and
damage. Predicting and/or preventing these outcomes is a primary
objective.

This analysis examines the damaging effects of severe weather conditions
(e.g.~hurricanes, tornadoes, thunderstorms, floods, etc.) on human
populations and the economy in the U.S. from 1950 to 2011.

As a result, the analysis will highlight the severe weather events
associated with the greatest impact on the economy and population
health.

\hypertarget{synopsis}{%
\subsection{SYNOPSIS}\label{synopsis}}

This is an exploration of the U.S. National Oceanic and Atmospheric
Administration's (NOAA) storm database. This database tracks
characteristics of major storms and weather events in the United States,
including when and where they occur, which type of event, as well as the
estimates of relevant fatalities, injuries, and various forms of damage.
The dataset used in this project is provided by the U.S. National
Oceanic and Atmospheric Administration (NOAA). This analysis discovered
that tornados are responsible for a maximum number of fatalities and
injuries. This analysis also discoered that floods are responsbile for
maximum property damage, while Droughts cause maximum crop damage.
Objective: Explore the NOAA Storm Database to help answer important
questions about severe weather events.

\hypertarget{data-processing}{%
\subsection{DATA PROCESSING}\label{data-processing}}

\begin{Shaded}
\begin{Highlighting}[]
\CommentTok{\# Load Libraries}
\FunctionTok{library}\NormalTok{(dplyr)}
\end{Highlighting}
\end{Shaded}

\begin{verbatim}
## 
## Attaching package: 'dplyr'
\end{verbatim}

\begin{verbatim}
## The following objects are masked from 'package:stats':
## 
##     filter, lag
\end{verbatim}

\begin{verbatim}
## The following objects are masked from 'package:base':
## 
##     intersect, setdiff, setequal, union
\end{verbatim}

\begin{Shaded}
\begin{Highlighting}[]
\CommentTok{\# Load Data}
\NormalTok{storm\_data }\OtherTok{\textless{}{-}} \FunctionTok{read.csv}\NormalTok{(}\StringTok{\textquotesingle{}repdata\_data\_StormData.csv\textquotesingle{}}\NormalTok{)}
\FunctionTok{head}\NormalTok{(storm\_data)}
\end{Highlighting}
\end{Shaded}

\begin{verbatim}
##   STATE__           BGN_DATE BGN_TIME TIME_ZONE COUNTY COUNTYNAME STATE  EVTYPE
## 1       1  4/18/1950 0:00:00     0130       CST     97     MOBILE    AL TORNADO
## 2       1  4/18/1950 0:00:00     0145       CST      3    BALDWIN    AL TORNADO
## 3       1  2/20/1951 0:00:00     1600       CST     57    FAYETTE    AL TORNADO
## 4       1   6/8/1951 0:00:00     0900       CST     89    MADISON    AL TORNADO
## 5       1 11/15/1951 0:00:00     1500       CST     43    CULLMAN    AL TORNADO
## 6       1 11/15/1951 0:00:00     2000       CST     77 LAUDERDALE    AL TORNADO
##   BGN_RANGE BGN_AZI BGN_LOCATI END_DATE END_TIME COUNTY_END COUNTYENDN
## 1         0                                               0         NA
## 2         0                                               0         NA
## 3         0                                               0         NA
## 4         0                                               0         NA
## 5         0                                               0         NA
## 6         0                                               0         NA
##   END_RANGE END_AZI END_LOCATI LENGTH WIDTH F MAG FATALITIES INJURIES PROPDMG
## 1         0                      14.0   100 3   0          0       15    25.0
## 2         0                       2.0   150 2   0          0        0     2.5
## 3         0                       0.1   123 2   0          0        2    25.0
## 4         0                       0.0   100 2   0          0        2     2.5
## 5         0                       0.0   150 2   0          0        2     2.5
## 6         0                       1.5   177 2   0          0        6     2.5
##   PROPDMGEXP CROPDMG CROPDMGEXP WFO STATEOFFIC ZONENAMES LATITUDE LONGITUDE
## 1          K       0                                         3040      8812
## 2          K       0                                         3042      8755
## 3          K       0                                         3340      8742
## 4          K       0                                         3458      8626
## 5          K       0                                         3412      8642
## 6          K       0                                         3450      8748
##   LATITUDE_E LONGITUDE_ REMARKS REFNUM
## 1       3051       8806              1
## 2          0          0              2
## 3          0          0              3
## 4          0          0              4
## 5          0          0              5
## 6          0          0              6
\end{verbatim}

\begin{Shaded}
\begin{Highlighting}[]
\FunctionTok{summary}\NormalTok{(storm\_data)}
\end{Highlighting}
\end{Shaded}

\begin{verbatim}
##     STATE__       BGN_DATE           BGN_TIME          TIME_ZONE        
##  Min.   : 1.0   Length:902297      Length:902297      Length:902297     
##  1st Qu.:19.0   Class :character   Class :character   Class :character  
##  Median :30.0   Mode  :character   Mode  :character   Mode  :character  
##  Mean   :31.2                                                           
##  3rd Qu.:45.0                                                           
##  Max.   :95.0                                                           
##                                                                         
##      COUNTY       COUNTYNAME           STATE              EVTYPE         
##  Min.   :  0.0   Length:902297      Length:902297      Length:902297     
##  1st Qu.: 31.0   Class :character   Class :character   Class :character  
##  Median : 75.0   Mode  :character   Mode  :character   Mode  :character  
##  Mean   :100.6                                                           
##  3rd Qu.:131.0                                                           
##  Max.   :873.0                                                           
##                                                                          
##    BGN_RANGE          BGN_AZI           BGN_LOCATI          END_DATE        
##  Min.   :   0.000   Length:902297      Length:902297      Length:902297     
##  1st Qu.:   0.000   Class :character   Class :character   Class :character  
##  Median :   0.000   Mode  :character   Mode  :character   Mode  :character  
##  Mean   :   1.484                                                           
##  3rd Qu.:   1.000                                                           
##  Max.   :3749.000                                                           
##                                                                             
##    END_TIME           COUNTY_END COUNTYENDN       END_RANGE       
##  Length:902297      Min.   :0    Mode:logical   Min.   :  0.0000  
##  Class :character   1st Qu.:0    NA's:902297    1st Qu.:  0.0000  
##  Mode  :character   Median :0                   Median :  0.0000  
##                     Mean   :0                   Mean   :  0.9862  
##                     3rd Qu.:0                   3rd Qu.:  0.0000  
##                     Max.   :0                   Max.   :925.0000  
##                                                                   
##    END_AZI           END_LOCATI            LENGTH              WIDTH         
##  Length:902297      Length:902297      Min.   :   0.0000   Min.   :   0.000  
##  Class :character   Class :character   1st Qu.:   0.0000   1st Qu.:   0.000  
##  Mode  :character   Mode  :character   Median :   0.0000   Median :   0.000  
##                                        Mean   :   0.2301   Mean   :   7.503  
##                                        3rd Qu.:   0.0000   3rd Qu.:   0.000  
##                                        Max.   :2315.0000   Max.   :4400.000  
##                                                                              
##        F               MAG            FATALITIES          INJURIES        
##  Min.   :0.0      Min.   :    0.0   Min.   :  0.0000   Min.   :   0.0000  
##  1st Qu.:0.0      1st Qu.:    0.0   1st Qu.:  0.0000   1st Qu.:   0.0000  
##  Median :1.0      Median :   50.0   Median :  0.0000   Median :   0.0000  
##  Mean   :0.9      Mean   :   46.9   Mean   :  0.0168   Mean   :   0.1557  
##  3rd Qu.:1.0      3rd Qu.:   75.0   3rd Qu.:  0.0000   3rd Qu.:   0.0000  
##  Max.   :5.0      Max.   :22000.0   Max.   :583.0000   Max.   :1700.0000  
##  NA's   :843563                                                           
##     PROPDMG         PROPDMGEXP           CROPDMG         CROPDMGEXP       
##  Min.   :   0.00   Length:902297      Min.   :  0.000   Length:902297     
##  1st Qu.:   0.00   Class :character   1st Qu.:  0.000   Class :character  
##  Median :   0.00   Mode  :character   Median :  0.000   Mode  :character  
##  Mean   :  12.06                      Mean   :  1.527                     
##  3rd Qu.:   0.50                      3rd Qu.:  0.000                     
##  Max.   :5000.00                      Max.   :990.000                     
##                                                                           
##      WFO             STATEOFFIC         ZONENAMES            LATITUDE   
##  Length:902297      Length:902297      Length:902297      Min.   :   0  
##  Class :character   Class :character   Class :character   1st Qu.:2802  
##  Mode  :character   Mode  :character   Mode  :character   Median :3540  
##                                                           Mean   :2875  
##                                                           3rd Qu.:4019  
##                                                           Max.   :9706  
##                                                           NA's   :47    
##    LONGITUDE        LATITUDE_E     LONGITUDE_       REMARKS         
##  Min.   :-14451   Min.   :   0   Min.   :-14455   Length:902297     
##  1st Qu.:  7247   1st Qu.:   0   1st Qu.:     0   Class :character  
##  Median :  8707   Median :   0   Median :     0   Mode  :character  
##  Mean   :  6940   Mean   :1452   Mean   :  3509                     
##  3rd Qu.:  9605   3rd Qu.:3549   3rd Qu.:  8735                     
##  Max.   : 17124   Max.   :9706   Max.   :106220                     
##                   NA's   :40                                        
##      REFNUM      
##  Min.   :     1  
##  1st Qu.:225575  
##  Median :451149  
##  Mean   :451149  
##  3rd Qu.:676723  
##  Max.   :902297  
## 
\end{verbatim}

\hypertarget{results}{%
\subsection{RESULTS}\label{results}}

\#\#\#QUESTION 1. Across the United States, which types of events (as
indicated in the EVTYPE variable) are most harmful with respect to
population health?

\begin{Shaded}
\begin{Highlighting}[]
\CommentTok{\#a) aggregating EVTYPE wrt injuries : aggregate the top 10 injuries by the event type and sort the output in descending order}
\NormalTok{total\_injuries }\OtherTok{\textless{}{-}} \FunctionTok{aggregate}\NormalTok{(INJURIES}\SpecialCharTok{\textasciitilde{}}\NormalTok{EVTYPE, storm\_data, sum)}
\NormalTok{total\_injuries }\OtherTok{\textless{}{-}} \FunctionTok{arrange}\NormalTok{(total\_injuries, }\FunctionTok{desc}\NormalTok{(INJURIES))}
\NormalTok{total\_injuries }\OtherTok{\textless{}{-}}\NormalTok{ total\_injuries[}\DecValTok{1}\SpecialCharTok{:}\DecValTok{20}\NormalTok{, ]}
\NormalTok{total\_injuries}
\end{Highlighting}
\end{Shaded}

\begin{verbatim}
##                EVTYPE INJURIES
## 1             TORNADO    91346
## 2           TSTM WIND     6957
## 3               FLOOD     6789
## 4      EXCESSIVE HEAT     6525
## 5           LIGHTNING     5230
## 6                HEAT     2100
## 7           ICE STORM     1975
## 8         FLASH FLOOD     1777
## 9   THUNDERSTORM WIND     1488
## 10               HAIL     1361
## 11       WINTER STORM     1321
## 12  HURRICANE/TYPHOON     1275
## 13          HIGH WIND     1137
## 14         HEAVY SNOW     1021
## 15           WILDFIRE      911
## 16 THUNDERSTORM WINDS      908
## 17           BLIZZARD      805
## 18                FOG      734
## 19   WILD/FOREST FIRE      545
## 20         DUST STORM      440
\end{verbatim}

\begin{Shaded}
\begin{Highlighting}[]
\CommentTok{\#b) aggregating EVTYPE wrt fatalities : aggregate the top 10 fatalities by the event type and sort the output in descending order}
\NormalTok{total\_fatalities }\OtherTok{\textless{}{-}} \FunctionTok{aggregate}\NormalTok{(FATALITIES}\SpecialCharTok{\textasciitilde{}}\NormalTok{EVTYPE,storm\_data, sum)}
\NormalTok{total\_fatalities }\OtherTok{\textless{}{-}} \FunctionTok{arrange}\NormalTok{(total\_fatalities, }\FunctionTok{desc}\NormalTok{(FATALITIES))}
\NormalTok{total\_fatalities }\OtherTok{\textless{}{-}}\NormalTok{ total\_fatalities[}\DecValTok{1}\SpecialCharTok{:}\DecValTok{20}\NormalTok{, ]}
\NormalTok{total\_fatalities}
\end{Highlighting}
\end{Shaded}

\begin{verbatim}
##                     EVTYPE FATALITIES
## 1                  TORNADO       5633
## 2           EXCESSIVE HEAT       1903
## 3              FLASH FLOOD        978
## 4                     HEAT        937
## 5                LIGHTNING        816
## 6                TSTM WIND        504
## 7                    FLOOD        470
## 8              RIP CURRENT        368
## 9                HIGH WIND        248
## 10               AVALANCHE        224
## 11            WINTER STORM        206
## 12            RIP CURRENTS        204
## 13               HEAT WAVE        172
## 14            EXTREME COLD        160
## 15       THUNDERSTORM WIND        133
## 16              HEAVY SNOW        127
## 17 EXTREME COLD/WIND CHILL        125
## 18             STRONG WIND        103
## 19                BLIZZARD        101
## 20               HIGH SURF        101
\end{verbatim}

Plotting the graph depicting the top 10 causes for Fatalities and
Injuries

\begin{Shaded}
\begin{Highlighting}[]
\CommentTok{\# plot graphs showing the top 10 fatalities and injuries}
\FunctionTok{par}\NormalTok{(}\AttributeTok{mfrow=}\FunctionTok{c}\NormalTok{(}\DecValTok{1}\NormalTok{,}\DecValTok{2}\NormalTok{),}\AttributeTok{mar=}\FunctionTok{c}\NormalTok{(}\DecValTok{10}\NormalTok{,}\DecValTok{3}\NormalTok{,}\DecValTok{3}\NormalTok{,}\DecValTok{2}\NormalTok{))}
\FunctionTok{barplot}\NormalTok{(total\_fatalities}\SpecialCharTok{$}\NormalTok{FATALITIES,}\AttributeTok{names.arg=}\NormalTok{total\_fatalities}\SpecialCharTok{$}\NormalTok{EVTYPE,}\AttributeTok{las=}\DecValTok{2}\NormalTok{,}\AttributeTok{col=}\StringTok{"maroon"}\NormalTok{,}\AttributeTok{ylab=}\StringTok{"fatalities"}\NormalTok{,}\AttributeTok{main=}\StringTok{"Top 10 fatalities"}\NormalTok{)}
\FunctionTok{barplot}\NormalTok{(total\_injuries}\SpecialCharTok{$}\NormalTok{INJURIES,}\AttributeTok{names.arg=}\NormalTok{total\_injuries}\SpecialCharTok{$}\NormalTok{EVTYPE,}\AttributeTok{las=}\DecValTok{2}\NormalTok{,}\AttributeTok{col=}\StringTok{"maroon"}\NormalTok{,}\AttributeTok{ylab=}\StringTok{"injuries"}\NormalTok{,}\AttributeTok{main=}\StringTok{"Top 10 Injuries"}\NormalTok{)}
\end{Highlighting}
\end{Shaded}

\includegraphics{Storm-Analysis_files/figure-latex/unnamed-chunk-4-1.pdf}

\#\#\#QUESTION 2. Across the United States, which types of events have
the greatest economic consequences? An analysis of the weather events
responsible for the greatest economic consequences

Hypothesis: Economic consequences means damages. The two significant
types of damage typically caused by weather events include `properties
and crops'

We have property Damage and crop damage

\begin{Shaded}
\begin{Highlighting}[]
\CommentTok{\# Aggregate Data for Property Damage}
\NormalTok{property\_damage }\OtherTok{\textless{}{-}} \FunctionTok{aggregate}\NormalTok{(PROPDMG }\SpecialCharTok{\textasciitilde{}}\NormalTok{ EVTYPE, }\AttributeTok{data =}\NormalTok{ storm\_data, }\AttributeTok{FUN =}\NormalTok{ sum)}
\NormalTok{property\_damage }\OtherTok{\textless{}{-}}\NormalTok{ property\_damage[}\FunctionTok{order}\NormalTok{(property\_damage}\SpecialCharTok{$}\NormalTok{PROPDMG, }\AttributeTok{decreasing =} \ConstantTok{TRUE}\NormalTok{), ]}
\CommentTok{\# 10 most harmful causes of injuries}
\NormalTok{property\_damageMax }\OtherTok{\textless{}{-}}\NormalTok{ property\_damage[}\DecValTok{1}\SpecialCharTok{:}\DecValTok{10}\NormalTok{, ]}
\FunctionTok{print}\NormalTok{(property\_damageMax)}
\end{Highlighting}
\end{Shaded}

\begin{verbatim}
##                 EVTYPE   PROPDMG
## 834            TORNADO 3212258.2
## 153        FLASH FLOOD 1420124.6
## 856          TSTM WIND 1335965.6
## 170              FLOOD  899938.5
## 760  THUNDERSTORM WIND  876844.2
## 244               HAIL  688693.4
## 464          LIGHTNING  603351.8
## 786 THUNDERSTORM WINDS  446293.2
## 359          HIGH WIND  324731.6
## 972       WINTER STORM  132720.6
\end{verbatim}

\begin{Shaded}
\begin{Highlighting}[]
\CommentTok{\#Aggregate Data for Crop Damage}
\NormalTok{crop\_damage }\OtherTok{\textless{}{-}} \FunctionTok{aggregate}\NormalTok{(CROPDMG }\SpecialCharTok{\textasciitilde{}}\NormalTok{ EVTYPE, }\AttributeTok{data =}\NormalTok{ storm\_data, }\AttributeTok{FUN =}\NormalTok{ sum)}
\NormalTok{crop\_damage }\OtherTok{\textless{}{-}}\NormalTok{ crop\_damage[}\FunctionTok{order}\NormalTok{(crop\_damage}\SpecialCharTok{$}\NormalTok{CROPDMG, }\AttributeTok{decreasing =} \ConstantTok{TRUE}\NormalTok{), ]}
\CommentTok{\# 10 most harmful causes of injuries}
\NormalTok{crop\_damageMax }\OtherTok{\textless{}{-}}\NormalTok{ crop\_damage[}\DecValTok{1}\SpecialCharTok{:}\DecValTok{10}\NormalTok{, ]}
\FunctionTok{print}\NormalTok{(crop\_damageMax)}
\end{Highlighting}
\end{Shaded}

\begin{verbatim}
##                 EVTYPE   CROPDMG
## 244               HAIL 579596.28
## 153        FLASH FLOOD 179200.46
## 170              FLOOD 168037.88
## 856          TSTM WIND 109202.60
## 834            TORNADO 100018.52
## 760  THUNDERSTORM WIND  66791.45
## 95             DROUGHT  33898.62
## 786 THUNDERSTORM WINDS  18684.93
## 359          HIGH WIND  17283.21
## 290         HEAVY RAIN  11122.80
\end{verbatim}

Plotting the graph depicting the top 10 causes for Property and Crop
Damage

\begin{Shaded}
\begin{Highlighting}[]
\DocumentationTok{\#\#plot the graph showing the top 10 property and crop damages}

\FunctionTok{par}\NormalTok{(}\AttributeTok{mfrow=}\FunctionTok{c}\NormalTok{(}\DecValTok{1}\NormalTok{,}\DecValTok{2}\NormalTok{),}\AttributeTok{mar=}\FunctionTok{c}\NormalTok{(}\DecValTok{11}\NormalTok{,}\DecValTok{3}\NormalTok{,}\DecValTok{3}\NormalTok{,}\DecValTok{2}\NormalTok{))}
\FunctionTok{barplot}\NormalTok{(property\_damageMax}\SpecialCharTok{$}\NormalTok{PROPDMG}\SpecialCharTok{/}\NormalTok{(}\DecValTok{10}\SpecialCharTok{\^{}}\DecValTok{5}\NormalTok{),}\AttributeTok{names.arg=}\NormalTok{property\_damageMax}\SpecialCharTok{$}\NormalTok{EVTYPE,}\AttributeTok{las=}\DecValTok{3}\NormalTok{,}\AttributeTok{col=}\StringTok{"green"}\NormalTok{,}\AttributeTok{ylab=}\StringTok{"Property damage(billions)"}\NormalTok{,}\AttributeTok{main=}\StringTok{"Top10 Property Damages"}\NormalTok{)}
\FunctionTok{barplot}\NormalTok{(crop\_damageMax}\SpecialCharTok{$}\NormalTok{CROPDMG}\SpecialCharTok{/}\NormalTok{(}\DecValTok{10}\SpecialCharTok{\^{}}\DecValTok{5}\NormalTok{),}\AttributeTok{names.arg=}\NormalTok{crop\_damageMax}\SpecialCharTok{$}\NormalTok{EVTYPE,}\AttributeTok{las=}\DecValTok{3}\NormalTok{,}\AttributeTok{col=}\StringTok{"green"}\NormalTok{,}\AttributeTok{ylab=}\StringTok{"Crop damage(billions)"}\NormalTok{,}\AttributeTok{main=}\StringTok{"Top10 Crop Damages"}\NormalTok{)}
\end{Highlighting}
\end{Shaded}

\includegraphics{Storm-Analysis_files/figure-latex/unnamed-chunk-7-1.pdf}

\hypertarget{summary}{%
\subsection{SUMMARY}\label{summary}}

Tornados are responsible for the maximum number of fatalities and
injuries, followed by Excessive Heat for fatalities and Thunderstorm
wind for injuries.

Floods are responsbile for maximum property damage, while Droughts cause
maximum crop damage. Second major events that caused the maximum damage
was Hurricanes/Typhoos for property damage and Floods for crop damage.

\end{document}
